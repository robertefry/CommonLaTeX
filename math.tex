
% Physics Packages
% \usepackage{siunitx} % broken with physics package
\usepackage{physics}
\usepackage{physics2}
\usepackage{physconst}

% Math Packages
\usepackage{mathtools}
\usepackage{amsmath}
% \usepackage{amsthm} (replaced with theorem::ntheorem)
\usepackage{amssymb}
\usepackage{amsfonts}

% ------------------------------------------------------------------------------
%   Enclosures
% ------------------------------------------------------------------------------

\NewDocumentCommand{\xenclose}{D<>{0}mmm}%
{%
    \begin{switch}{#1}%
        \xcase{0}{ \left#2 #4 \right#3 }%
        \xcase{1}{ \bigl#2 #4 \bigr#3 }%
        \xcase{2}{ \Bigl#2 #4 \Bigr#3 }%
        \xcase{3}{ \biggl#2 #4 \biggr#3 }%
        \xcase{4}{ \Biggl#2 #4 \Biggr#3 }%
        \xcase{}{ #2 #4 #3 }%
    \end{switch}%
}%

\NewDocumentCommand{\xboxed}{D<>{0}m}%
{%
    \boxed{ \xenclose<#1>{.}{.}{#2} }%
}%

% ~ Named Enclosures ~~~~~~~~~~~~~~~~~~~~~~~~~~~~~~~~~~~~~~~~~~~~~~~~~~~~~~~~~~~

\NewMap{xopeners} \MapSet{xopeners}{.}{.}
\NewMap{xclosers} \MapSet{xclosers}{.}{.}

\NewDocumentCommand{\DeclareNamedEnclosure}{>{\SplitList{,}}m mm}
{
    \newcommand{\DefineThisNamedEnclosure}[1]
    {
        \expandafter\DeclareDocumentCommand\csname x##1\endcsname{D<>{0}m}%
        {%
            \xenclose<####1>{#2}{#3}{####2}%
        }%
        \MapSet{xopeners}{##1}{#2}
        \MapSet{xclosers}{##1}{#3}
    }
    \ProcessList{#1}{\DefineThisNamedEnclosure}
    \let\DefineThisNamedEnclosure\undefined
}

\DeclareNamedEnclosure{p,paren}     {(}{)}
\DeclareNamedEnclosure{b,brack}     {[}{]}
\DeclareNamedEnclosure{s,brace,set} {\{}{\}}
\DeclareNamedEnclosure{g,chev,gen}  {\langle}{\rangle}
\DeclareNamedEnclosure{a,abs}       {\vert}{\vert}
\DeclareNamedEnclosure{n,norm}      {\Vert}{\Vert}
\DeclareNamedEnclosure{  floor}     {\lfloor}{\rfloor}
\DeclareNamedEnclosure{  ceil}      {\lceil}{\rceil}
\DeclareNamedEnclosure{  round}     {\lfloor}{\rceil}

% ~ Interval Enclosure ~~~~~~~~~~~~~~~~~~~~~~~~~~~~~~~~~~~~~~~~~~~~~~~~~~~~~~~~~

\NewDocumentCommand{\xinterval}{D<>{0}mm}%
{%
    \begin{switch}{#2}%
        \xcase{o} { \xenclose<#1>{(}{)}{#3} }%
        \xcase{c} { \xenclose<#1>{[}{]}{#3} }%
        \xcase{oo}{ \xenclose<#1>{(}{)}{#3} }%
        \xcase{cc}{ \xenclose<#1>{[}{]}{#3} }%
        \xcase{oc}{ \xenclose<#1>{(}{]}{#3} }%
        \xcase{co}{ \xenclose<#1>{[}{)}{#3} }%
    \end{switch}%
}%

\NewDocumentCommand{\xio}{D<>{0}m}{ \xinterval<#1>{o}{#2} }
\NewDocumentCommand{\xic}{D<>{0}m}{ \xinterval<#1>{c}{#2} }
\NewDocumentCommand{\xioo}{D<>{0}m}{ \xinterval<#1>{oo}{#2} }
\NewDocumentCommand{\xioc}{D<>{0}m}{ \xinterval<#1>{oc}{#2} }
\NewDocumentCommand{\xico}{D<>{0}m}{ \xinterval<#1>{co}{#2} }
\NewDocumentCommand{\xicc}{D<>{0}m}{ \xinterval<#1>{cc}{#2} }

% ~ Other Enclosures ~~~~~~~~~~~~~~~~~~~~~~~~~~~~~~~~~~~~~~~~~~~~~~~~~~~~~~~~~~~

% TODO: Fix D<>{0} argument parsing for \xgroup

\NewDocumentCommand{\xgroup}{D<>{0}O{.}m}%
{%
    \xenclose<#1>{\MapGet{xopeners}{#2}}{\MapGet{xclosers}{#2}}{#3}%
}%

% Functions
\NewDocumentCommand{\xf}{E{_^}{} D<>{0}O{.}mE{_^}{}}%
{%
    \ifel{\IfValueT{#1}{.}\IfValueT{#2}{.}}{}%
    {%
        \xgroup<#3>[#4]{#5}%
    }{%
        \prescript%
            {\IfValueT{#2}{#2}}%
            {\IfValueT{#1}{#1}}%
        {\xgroup<#3>[#4]{#5}}%
    }%
        \IfValueT{#6}{_{#6}}%
        \IfValueT{#7}{^{#7}}%
    %
    \xfrecurse%     % recursive step
}%
\NewDocumentCommand{\xfrecurse}{D<>{0}O{p}ge{_^}}%
{%
    \IfValueT{#3}%
    {%
        % \let\close%
        \xgroup<#1>[#2]{#3}%
            \IfValueT{#4}{_{#4}}%
            \IfValueT{#5}{^{#5}}%
        %
        \xfrecurse%     % recursive step
    }%
}%

% ------------------------------------------------------------------------------
%   Symbols
% ------------------------------------------------------------------------------

\usepackage[g]{esvect}
\NewDocumentCommand{\xv}{m}{ \vv{#1} }

\NewDocumentCommand{\xm}{O{p}m}{ \xgroup[#1]{\begin{matrix}#2\end{matrix}} }

\NewDocumentCommand{\pd}{O{}m}{\partial^{#1} #2}

% ------------------------------------------------------------------------------
%   Operators
% ------------------------------------------------------------------------------

\renewcommand{\DeclareMathOperator}[2]%
{%
    \expandafter\DeclareDocumentCommand{#1}{e{_^}D<>{0}O{p}g}%
    {%
        \operatorname{\textnormal{#2}}\IfValueT{##1}{_{##1}}\IfValueT{##2}{^{##2}}%
        \IfValueT{##5}{\xgroup<##3>[##4]{##5}}%
    }%
}%

\DeclareMathOperator{\Id}{Id}           % identity

\DeclareMathOperator{\Hom}{Hom}         % homomorphism
\DeclareMathOperator{\Iso}{Iso}         % isomorphism
\DeclareMathOperator{\Aut}{Aut}         % automorphism
\DeclareMathOperator{\Mon}{Mon}         % monomorphism
\DeclareMathOperator{\Epi}{Epi}         % epimorphism

\DeclareMathOperator{\Ker}{Ker}         % kernel
\DeclareMathOperator{\Dom}{Dom}         % domain
\DeclareMathOperator{\CoDom}{CoDom}     % co-domain
\DeclareMathOperator{\Img}{Im}          % image
\DeclareMathOperator{\Ann}{Ann}         % annihilator

\DeclareMathOperator{\Mat}{Mat}         % matrix
\DeclareMathOperator{\GL}{GL}           % general linear group

\DeclareMathOperator{\Cat}{Cat}         % category
\DeclareMathOperator{\Set}{Set}         % set
\DeclareMathOperator{\Grp}{Grp}         % group
\DeclareMathOperator{\Ring}{Ring}       % ring
\DeclareMathOperator{\Top}{Top}         % topology
\DeclareMathOperator{\Met}{Met}         % metric space
\DeclareMathOperator{\Vect}{Vect}       % vector space
\DeclareMathOperator{\Mod}{Mod}         % module
