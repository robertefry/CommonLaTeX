
% ------------------------------------------------------------------------------
%   Package Includes
% ------------------------------------------------------------------------------

% Programming
\usepackage{xparse}
\usepackage{etoolbox}

% Dummy Text
\usepackage[english]{babel}
\usepackage{lipsum}
\usepackage{blindtext}

% Page Formatting
\usepackage[many]{tcolorbox}
\usepackage{vwcol}
\usepackage{multicol}
\usepackage{wrapfig}

% Blackboard Fonts
\usepackage{bbold}

% Extra Symbols
\usepackage{cancel}
\usepackage{wasysym}
\usepackage{stmaryrd}

% Text Formatting
\usepackage{exercise}
\usepackage[inline,shortlabels]{enumitem}
\usepackage{listings}
\usepackage{siunitx}
\usepackage{xcolor}

% Math Formatting
\usepackage{stackrel}

% Tikz Picture
\usepackage{tikz}
\usepackage{circuitikz}
\usepackage{tkz-euclide}
\usetikzlibrary{calc}
\usetikzlibrary{positioning}
\usetikzlibrary{angles,quotes}
\usetikzlibrary{shapes.geometric,arrows}

% Graph Plotting
\usepackage{pgfplots}
\pgfplotsset{compat=1.17}

% Physics Packages
\usepackage{physics}
\usepackage{physconst}

% Math Packages
\usepackage{mathtools}
\usepackage{amsmath}
\usepackage{amsthm}
\usepackage{amssymb}
\usepackage{amsfonts}


% ------------------------------------------------------------------------------
%   Un-named Character Sets
% ------------------------------------------------------------------------------

% Un-named Greek Characters
\newcommand{\omicron}{o}
\newcommand{\Alpha}{A}
\newcommand{\Beta}{B}
\newcommand{\Epsilon}{E}
\newcommand{\Zeta}{Z}
\newcommand{\Eta}{H}
\newcommand{\Iota}{I}
\newcommand{\Kappa}{K}
\newcommand{\Mu}{M}
\newcommand{\Nu}{N}
\newcommand{\Omicron}{O}
\newcommand{\Rho}{P}
\newcommand{\Tau}{T}
\newcommand{\Chi}{X}


% ------------------------------------------------------------------------------
%   Programming
% ------------------------------------------------------------------------------

% IF Statement
\usepackage{xifthen}
\newcommand{\ifeq}[3]{\ifthenelse{\equal{#1}{#2}}{#3}{  }}
\newcommand{\ifne}[3]{\ifthenelse{\equal{#1}{#2}}{  }{#3}}
\newcommand{\ifel}[4]{\ifthenelse{\equal{#1}{#2}}{#3}{#4}}

% SWITCH Statement
\newcommand{\xcase}[2]{#1 #2} % Dummy, so \renewcommand has something to overwrite...
\NewDocumentEnvironment{switch}{m}{\renewcommand{\xcase}{\ifeq{#1}}}{}


% ------------------------------------------------------------------------------
%   Page Formatting
% ------------------------------------------------------------------------------

% Break Commands
\newcommand{\xbreakpage}{\mbox{}\newpage}
\newcommand{\xbreakparg}{\mbox{}\par}
\newcommand{\xbreakline}{\mbox{}\newline}

% Force GoodBreak Between Sections (redundant?)
\usepackage{titlesec}
\newcommand{\sectionbreak}{\goodbreak}
\newcommand{\subsectionbreak}{\goodbreak}
\newcommand{\subsubsectionbreak}{\goodbreak}


% ------------------------------------------------------------------------------
%   Text Commands
% ------------------------------------------------------------------------------

% Repeat a String N Times
\makeatletter
\newcount\x@repeat@count
\newcommand{\xrepeat}[2]{%
  \begingroup
  \x@repeat@count=\z@
  \@whilenum\x@repeat@count<#1\do{#2\advance\x@repeat@count\@ne}%
  \endgroup
}
\makeatother

% Format TODO Marker
\NewDocumentCommand{\TODO}{}
{
    \begin{center}
        \color{red}
        \textbf{\textit{\xrepeat{6}{=}\underline{TODO}\xrepeat{6}{=}}}
    \end{center}
}

% Block Environments
\NewDocumentEnvironment{block}{D<>{0em}O{m}O{}}
{
    \begin{center}

    \ifeq{#2}{m}
    {
        \begin{tcolorbox}[breakable,enhanced,width=\linewidth-#1,blanker]
        \ifne{#3}{}{ \textit{\underline{#3:}}\hspace{\parindent} }
    }
    \ifeq{#2}{M}
    {
        \begin{tcolorbox}[breakable,enhanced,width=\linewidth-#1,sharp corners,boxrule=1pt]
        \ifne{#3}{}{ \textit{\underline{#3:}}\hspace{\parindent} }
    }
    \ifeq{#2}{t}
    {
        \begin{tcolorbox}[breakable,enhanced,width=\linewidth-#1,blanker]
        \ifne{#3}{}{ \textbf{#3}\vspace{0.33\baselineskip}\xbreakline{} }
    }
    \ifeq{#2}{T}
    {
        \ifel{#3}{}
        {
            \begin{tcolorbox}[breakable,enhanced,width=\linewidth-#1,sharp corners,boxrule=1pt]
        }{
            \begin{tcolorbox}[breakable,enhanced,width=\linewidth-#1,title=\textbf{#3},sharp corners,boxrule=1pt]
        }
    }
    \ifeq{#2}{C}
    {
        \ifel{#3}{}
        {
            \begin{tcolorbox}[breakable,enhanced,width=\linewidth-#1,sharp corners,boxrule=1pt]
        }{
            \begin{tcolorbox}[breakable,enhanced,width=\linewidth-#1,title=\textbf{#3},sharp corners,boxrule=1pt]
        }
        \begin{quote}
    }
}{
    \ifeq{#2}{C}
    {
        \end{quote}
    }

    \end{tcolorbox}
    \end{center}
}

% Named Block Environments
\NewDocumentEnvironment{claim}    {D<>{0em}}{ \begin{block}[m][Claim]     }{ \end{block} }
\NewDocumentEnvironment{condition}{D<>{0em}}{ \begin{block}[m][Condition] }{ \end{block} }
\NewDocumentEnvironment{case}     {D<>{0em}}{ \begin{block}[m][Case]      }{ \end{block} }
\NewDocumentEnvironment{remark}   {D<>{0em}}{ \begin{block}[m][Remark]    }{ \end{block} }
\NewDocumentEnvironment{exempli}  {D<>{0em}}{ \begin{block}[m][Example]   }{ \end{block} }
\NewDocumentEnvironment{note}     {D<>{0em}}{ \begin{block}[m][Note]      }{ \end{block} }
\NewDocumentEnvironment{notice}   {D<>{0em}}{ \begin{block}[m][Notice]    }{ \end{block} }
\NewDocumentEnvironment{hint}     {D<>{0em}}{ \begin{block}[m][Hint]      }{ \end{block} }
\NewDocumentEnvironment{todo}     {D<>{0em}}{ \begin{block}[m][TODO]      }{ \end{block} }

% Theorem Environments
\newtheorem{counter}{counter}[subsection]
\NewDocumentCommand{\DefineMathEnvironment}{mmO{\itshape}}
{
    \newtheoremstyle{thmstyle}{\topsep}{\topsep}{#3}{}{\bfseries}{}{\newline}{}
    \theoremstyle{thmstyle}
    \newtheorem{#1}[counter]{#2}
    \AfterEndEnvironment{#1}{\goodbreak}
}
\DefineMathEnvironment{definition}{Definition}
\DefineMathEnvironment{axiom}{Axiom}
\DefineMathEnvironment{postulate}{Postulate}
\DefineMathEnvironment{lemma}{Lemma}
\DefineMathEnvironment{theorem}{Theorem}
\DefineMathEnvironment{corollary}{Corollary}
\DefineMathEnvironment{example}{Example}[]


% ------------------------------------------------------------------------------
%   Math Commands
% ------------------------------------------------------------------------------

% Vector Commands
\usepackage[g]{esvect}
\NewDocumentCommand{\xv}{m}{ \vv{#1} }

% Matrix Commands
\NewDocumentCommand{\xm}{O{p}m}{ \begin{#1matrix} #2 \end{#1matrix} }

% Math Symbols
\NewDocumentCommand{\divs}{}{ \;\big\vert\; }
\NewDocumentCommand{\nmapsto}{}{ \;\cancel{\mapsto}\; }
\NewDocumentCommand{\nequiv}{}{ \;\cancel{\equiv}\; }

% Math Commands
\NewDocumentCommand{\usb}{mm}{\underset{#1}{\underbrace{#2}}}
\NewDocumentCommand{\osb}{mm}{\overset{#1}{\overbrace{#2}}}

% ~ Math Enclosures ~~~~~~~~~~~~~~~~~~~~~~~~~~~~~~~~~~~~~~~~~~~~~~~~~~~~~~~~~~~~

% Math Enclosures
\NewDocumentCommand{\xenclose}{D<>{0}mmm}
{
    \begin{switch}{#1}
        \xcase{0}{ \left#2 #4 \right#3 }
        \xcase{1}{ \bigl#2 #4 \bigr#3 }
        \xcase{2}{ \Bigl#2 #4 \Bigr#3 }
        \xcase{3}{ \biggl#2 #4 \biggr#3 }
        \xcase{4}{ \Biggl#2 #4 \Biggr#3 }
        \xcase{-1}{ #2 #4 #3 }
    \end{switch}
}

% Math Enclosure Groups
\NewDocumentCommand{\xgroup}{D<>{0}O{.}m}
{
    \begin{switch}{#2}
        \xcase{.}{ \xenclose<#1>{.}{.}{#3} }             % default
        \xcase{p}{ \xenclose<#1>{(}{)}{#3} }             % parenthesis
        \xcase{b}{ \xenclose<#1>{[}{]}{#3} }             % brackets
        \xcase{B}{ \xenclose<#1>{\{}{\}}{#3} }           % braces
        \xcase{c}{ \xenclose<#1>{\langle}{\rangle}{#3} } % chevrons
        \xcase{v}{ \xenclose<#1>{\vert}{\vert}{#3} }     % absolute
        \xcase{V}{ \xenclose<#1>{\Vert}{\Vert}{#3} }     % normal
        \xcase{f}{ \xenclose<#1>{\lfloor}{\rfloor}{#3} } % floor
        \xcase{F}{ \xenclose<#1>{\lceil}{\rceil}{#3} }   % ceil
        \xcase{r}{ \xenclose<#1>{\lfloor}{\rceil}{#3} }  % round
    \end{switch}
}

% Named Enclosure Groups
\NewDocumentCommand{\xparen}{D<>{0}m}{ \xgroup<#1>[p]{#2} }
\NewDocumentCommand{\xbrack}{D<>{0}m}{ \xgroup<#1>[b]{#2} }
\NewDocumentCommand{\xbrace}{D<>{0}m}{ \xgroup<#1>[B]{#2} }
\NewDocumentCommand{\xchevn}{D<>{0}m}{ \xgroup<#1>[c]{#2} }
\NewDocumentCommand{\xabs}  {D<>{0}m}{ \xgroup<#1>[v]{#2} }
\NewDocumentCommand{\xnorm} {D<>{0}m}{ \xgroup<#1>[V]{#2} }
\NewDocumentCommand{\xfloor}{D<>{0}m}{ \xgroup<#1>[f]{#2} }
\NewDocumentCommand{\xceil} {D<>{0}m}{ \xgroup<#1>[F]{#2} }
\NewDocumentCommand{\xround}{D<>{0}m}{ \xgroup<#1>[r]{#2} }

% Math Function Enclosures
\NewDocumentCommand{\xfunc}{D<>{0}O{f}mm}
{
    \begin{switch}{#2}
        \xcase{f}{ #3\xparen<#1>{#4} } % function
        \xcase{t}{ #3\xbrace<#1>{#4} } % transform
        \xcase{a}{ #3\xbrack<#1>{#4} } % adjoin
    \end{switch}
}

% Named Function Enclosures
%\NewDocumentCommand{\xfunc}  {D<>{0}mm}{ \xfunc<#1>[f]{#2}{#3} }
\NewDocumentCommand{\xtrans} {D<>{0}mm}{ \xfunc<#1>[t]{#2}{#3} }
\NewDocumentCommand{\xadjoin}{D<>{0}mm}{ \xfunc<#1>[a]{#2}{#3} }


% ------------------------------------------------------------------------------
%   Tikz Styles
% ------------------------------------------------------------------------------

\tikzstyle{flow_startstop} = [rectangle,rounded corners,
minimum width=3cm,minimum height=1cm,text centered,draw=black]
\tikzstyle{flow_inout} = [trapezium,trapezium left angle=70,trapezium right angle=110,
minimum width=3cm,minimum height=1cm,text centered,draw=black]
\tikzstyle{flow_process} = [rectangle
minimum width=3cm,minimum height=1cm,text centered,draw=black]
\tikzstyle{flow_decision} = [diamond
minimum width=3cm,minimum height=1cm,text centered,draw=black]

\tikzstyle{arrow} = [thick,->,>=stealth]
\tikzstyle{process} = [rectangle,minimum width=3cm,minimum height=1cm,text centered,draw=black]


% ------------------------------------------------------------------------------
%   Source Code
% ------------------------------------------------------------------------------

\usepackage{sourcecodepro}
\definecolor{color_keyword}{rgb}{0.500,0.000,0.125}
\definecolor{color_comment}{rgb}{0.500,0.500,0.500}
\definecolor{color_strings}{rgb}{0.000,0.500,0.250}

\lstdefinelanguage{PlainText}{
}
\lstset{
    basicstyle=\scriptsize{}\ttfamily{},
    keywordstyle=\color{color_keyword},
    commentstyle=\color{color_comment},
    stringstyle=\color{color_strings},
    numbers=left,
    numberstyle=\tiny{},
    tabsize=4,
    breakatwhitespace=false,
    breaklines=true,
    showtabs=false,
    tabsize=4,
}

\NewDocumentEnvironment{sourcecode}{D<>{0em}O{Source Code}}
{
    \begin{block}<#2>[C][#2]
}{
    \end{block}
}
