
\usepackage{xparse}
\usepackage[thmmarks]{ntheorem}

% Redefine the `\thmcounter` command to modify the primary theorem counter.
\ProvideDocumentCommand{\thmcounter}{}{section}

% ------------------------------------------------------------------------------
%   Theorem Environments
% ------------------------------------------------------------------------------

% This command creates a new theorem environment.
%
% {#1}: Name of the new theorem environment.    (mandatory)
% [#2]: Name of a shared counter.               (optional)
% {#3}: Theorem title.                          (optional, defaults to #1 if not provided)
% [#4]: Name of a parent counter.               (optional)
%
\NewDocumentCommand{\MakeNewTheorem}{momo}
{
    \IfValueTF{#4}
    {
        \newtheorem{#1}{#3}[#4]
    }{
    \IfValueTF{#2}
    {
        \newtheorem{#1}[#2]{#3}
    }{
        \newtheorem{#1}{#3}
    }}
}

% todo: indent the theorem counter to match the chapter, section, etc.
%  * append code to the \chapter, \section, \subsection, \subsubsection commands
%  * to avoid theorems and sections with the same number, the section counter
%    should inherit from the theorem counter
%  * define the \endsection, \endsubsection, etc. commands which revert the
%    counters.
%
% A dummy environment so the actual theorems can use the same `thm' counter.
\newtheorem{thm}{thm}[\thmcounter]

\theoreminframepreskip{0pt}
\theoreminframepostskip{0pt}
\NewDocumentCommand{\qedhere}{O{\proofSymbol}}{\hfill#1}

% ~ Proof Environment ~~~~~~~~~~~~~~~~~~~~~~~~~~~~~~~~~~~~~~~~~~~~~~~~~~~~~~~~~~

{
    \theoremstyle{nonumberplain}
    \theoremheaderfont{\itshape}
    \theorembodyfont{\upshape}
    \theoremseparator{.}
    \theoremsymbol{\ensuremath{\square}}

    \MakeNewTheorem{proof}{Proof}
}

% ~ Minor Theorem Environments ~~~~~~~~~~~~~~~~~~~~~~~~~~~~~~~~~~~~~~~~~~~~~~~~~

\makeatletter\newtheoremstyle{blockminor}
    {\item[\hskip\labelsep\theorem@headerfont##1\theorem@separator]}
    {\item[\hskip\labelsep\theorem@headerfont##1\ (##3)\theorem@separator]}
\makeatother

{
    \theoremstyle{blockminor}
    \theoremheaderfont{\bfseries\scshape}
    \theorembodyfont{\upshape}
    \theoremseparator{:}
    \theoremsymbol{}

    \MakeNewTheorem{claim}{Claim}
    \MakeNewTheorem{condition}{Condition}
    \MakeNewTheorem{case}{Case}
    \MakeNewTheorem{remark}{Remark}
    \MakeNewTheorem{thatis}{That is}
    \MakeNewTheorem{exempli}{Example}
    \MakeNewTheorem{note}{Note}
    \MakeNewTheorem{notice}{Notice}
    \MakeNewTheorem{hint}{Hint}
    \MakeNewTheorem{todo}{TODO}
}

% ~ Major Theorem Environments ~~~~~~~~~~~~~~~~~~~~~~~~~~~~~~~~~~~~~~~~~~~~~~~~~

\NewDocumentCommand{\MakeMajorTheorem}{momo}
{
    \MakeNewTheorem{#1}[#2]{#3}[#4]
    \BeforeBeginEnvironment{#1}{\penalty-500}
    \BeforeBeginEnvironment{#1*}{\penalty-500}
}

\makeatletter\newtheoremstyle{blockmajor}
    {\item[\hskip\labelsep\theorem@headerfont ##1\ ##2\theorem@separator]}
    {\item[\hskip\labelsep\theorem@headerfont ##1\ ##2{\normalfont\ (##3)}\theorem@separator]}
\makeatother

{
    \theoremstyle{blockmajor}
    \theoremheaderfont{\normalfont\bfseries}
    \theorembodyfont{\upshape}
    \theoremseparator{.}
    \theoremsymbol{}

    \MakeMajorTheorem{definition}[thm]{Definition}
    \MakeMajorTheorem{proposition}[thm]{Proposition}
    \MakeMajorTheorem{postulate}[thm]{Postulate}
    \MakeMajorTheorem{lemma}[thm]{Lemma}
    \MakeMajorTheorem{theorem}[thm]{Theorem}
    \MakeMajorTheorem{corollary}[thm]{Corollary}
}

% ~ Titled Theorem Environments ~~~~~~~~~~~~~~~~~~~~~~~~~~~~~~~~~~~~~~~~~~~~~

\makeatletter\newtheoremstyle{blocktitle}
    {\item[\rlap{\vbox{\hbox{\hskip\labelsep\theorem@headerfont ##1\ ##2\theorem@separator}\hbox{\strut}}}]}
    {\item[\rlap{\vbox{\hbox{\hskip\labelsep\theorem@headerfont ##1\ ##2{\normalfont\ (##3)}\theorem@separator}\hbox{\strut}}}]}
\makeatother

{
    \theoremstyle{blockmajor}
    \theoremheaderfont{\normalfont\bfseries}
    \theorembodyfont{\upshape}

    \MakeMajorTheorem{example}[thm]{Example}
}
