
\usepackage{xparse}
\usepackage[thmmarks]{ntheorem}

% Redefine the `\thmcounter` command to modify the primary theorem counter.
\ProvideDocumentCommand{\thmcounter}{}{section}

% ------------------------------------------------------------------------------
%   Block Environment
% ------------------------------------------------------------------------------

% Define a new document environment called 'block'
%
% <#1> Indentation from the margins.    (optional, default 0em)
% [#2] Style of the block.              (optional, default 'm')
% [#3] Title or label for the block.    (optional)
%
%  - Style 'm': Minimal block with blanker style
%  - Style 'M': Minimal block with sharp corners and visible border
%  - Style 't': Title block with blanker style
%  - Style 'T': Title block with sharp corners and visible border
%  - Style 's': Titled box with sharp corners and visible border
%
\NewDocumentEnvironment{block}{D<>{0em}O{m}o}%
{%
    \begin{center}%
    \ifeq{#2}{m}%
    {%
        \begin{tcolorbox}[breakable,enhanced,width=\linewidth-#1,blanker]%
        \IfValueT{#3}{\textsl{\underline{#3:}\ }}%
    }%
    \ifeq{#2}{M}%
    {%
        \begin{tcolorbox}[breakable,enhanced,width=\linewidth-#1,sharp corners,boxrule=1pt]%
        \IfValueT{#3}{\textsl{\underline{#3:}\ }}%
    }%
    \ifeq{#2}{t}%
    {%
        \begin{tcolorbox}[breakable,enhanced,width=\linewidth-#1,blanker]%
        \IfValueT{#3}{\textbf{#3}\xbreakline}%
    }%
    \ifeq{#2}{T}%
    {%
        \begin{tcolorbox}[breakable,enhanced,width=\linewidth-#1,sharp corners,boxrule=1pt]%
        \IfValueT{#3}{\textbf{#3}\xbreakline}%
    }%
    \ifeq{#2}{s}%
    {%
        \IfValueTF{#3}%
            {\begin{tcolorbox}[breakable,enhanced,width=\linewidth-#1,sharp corners,boxrule=1pt,title=\textbf{#3}]}%
            {\begin{tcolorbox}[breakable,enhanced,width=\linewidth-#1,sharp corners,boxrule=1pt]}%
    }%
}{%
    \end{tcolorbox}%
    \end{center}%
}%

% Block Lists
\NewDocumentEnvironment{*itemize}{}{\begin{itemize}}{\end{itemize}}
\NewDocumentEnvironment{*itemize*}{}{\begin{block}<3em>\begin{itemize*}}{\end{itemize*}\end{block}}
\NewDocumentEnvironment{*enumerate}{}{\begin{enumerate}}{\end{enumerate}}
\NewDocumentEnvironment{*enumerate*}{}{\begin{block}<3em>\begin{enumerate*}}{\end{enumerate*}\end{block}}

% ------------------------------------------------------------------------------
%   Theorem Environments
% ------------------------------------------------------------------------------

\theoreminframepreskip{0pt}
\theoreminframepostskip{0pt}
\NewDocumentCommand{\qedhere}{O{\proofSymbol}}{\hfill#1}

% This command creates a new theorem environment.
% It also creates a starred version of the environment that is unnumbered and wrapped in a 'block' environment.
%
% {#1}: Name of the new theorem environment.    (mandatory)
% [#2]: Name of a shared counter.               (optional)
% {#3}: Theorem title.                          (optional, defaults to #1 if not provided)
% [#4]: Name of a parent counter.               (optional)
%
\NewDocumentCommand{\MakeNewTheorem}{momo}%
{%
    \IfValueTF{#4}
    {%
        \newtheorem{#1}{#3}[#4]%
    }{%
    \IfValueTF{#2}%
    {%
        \newtheorem{#1}[#2]{#3}%
    }{%
        \newtheorem{#1}{#3}%
    }}%
    % Create a starred version of the theorem environment.
    % This version is unnumbered and wrapped in a 'block' environment.
    \NewDocumentEnvironment{*#1}{}{\begin{block}[M]\begin{#1}}{\end{#1}\end{block}}
}%

% A dummy environment so the actual theorems can use the same `thm' counter.
\newtheorem{thm}{thm}[\thmcounter]

% ~ Proof Environment ~~~~~~~~~~~~~~~~~~~~~~~~~~~~~~~~~~~~~~~~~~~~~~~~~~~~~~~~~~

{
    \theoremstyle{nonumberplain}
    \theoremheaderfont{\itshape}
    \theorembodyfont{\upshape}
    \theoremseparator{.}
    \theoremsymbol{\ensuremath{\square}}

    \MakeNewTheorem{proof}{Proof}
}

% ~ Minor Theorem Environments ~~~~~~~~~~~~~~~~~~~~~~~~~~~~~~~~~~~~~~~~~~~~~~~~~

\makeatletter\newtheoremstyle{blockminor}%
    {\item[\hskip\labelsep\theorem@headerfont\underline{##1}\theorem@separator]}%
    {\item[\hskip\labelsep\theorem@headerfont\underline{##1}\ (##3)\theorem@separator]}%
\makeatother%

{
    \theoremstyle{blockminor}
    \theoremheaderfont{\itshape}
    \theorembodyfont{\upshape}
    \theoremseparator{;}
    \theoremsymbol{}

    \MakeNewTheorem{claim}{Claim}
    \MakeNewTheorem{condition}{Condition}
    \MakeNewTheorem{case}{Case}
    \MakeNewTheorem{remark}{Remark}
    \MakeNewTheorem{thatis}{That is}
    \MakeNewTheorem{exempli}{Example}
    \MakeNewTheorem{note}{Note}
    \MakeNewTheorem{notice}{Notice}
    \MakeNewTheorem{hint}{Hint}
    \MakeNewTheorem{todo}{TODO}
}

% ~ Major Theorem Environments ~~~~~~~~~~~~~~~~~~~~~~~~~~~~~~~~~~~~~~~~~~~~~~~~~

\NewDocumentCommand{\MakeMajorTheorem}{momo}%
{%
    \MakeNewTheorem{#1}[#2]{#3}[#4]%
    \BeforeBeginEnvironment{#1}{\penalty-500}%
    \BeforeBeginEnvironment{#1*}{\penalty-500}%
}%

\makeatletter\newtheoremstyle{blockmajor}%
    {\item[\hskip\labelsep\theorem@headerfont ##1\ ##2\theorem@separator]}%
    {\item[\hskip\labelsep\theorem@headerfont ##1\ ##2{\normalfont\ (##3)}\theorem@separator]}%
\makeatother%

{
    \theoremstyle{blockmajor}
    \theoremheaderfont{\normalfont\bfseries}
    \theorembodyfont{\slshape}
    \theoremseparator{.}
    \theoremsymbol{}

    \MakeMajorTheorem{definition}[thm]{Definition}
    \MakeMajorTheorem{proposition}[thm]{Proposition}
    \MakeMajorTheorem{postulate}[thm]{Postulate}
    \MakeMajorTheorem{lemma}[thm]{Lemma}
    \MakeMajorTheorem{theorem}[thm]{Theorem}
    \MakeMajorTheorem{corollary}[thm]{Corollary}
}

% ~ Titled Theorem Environments ~~~~~~~~~~~~~~~~~~~~~~~~~~~~~~~~~~~~~~~~~~~~~

\makeatletter\newtheoremstyle{blocktitle}%
    {\item[\rlap{\vbox{\hbox{\hskip\labelsep\theorem@headerfont ##1\ ##2\theorem@separator}\hbox{\strut}}}]}%
    {\item[\rlap{\vbox{\hbox{\hskip\labelsep\theorem@headerfont ##1\ ##2{\normalfont\ (##3)}\theorem@separator}\hbox{\strut}}}]}%
\makeatother%

{
    \theoremstyle{blockmajor}
    \theoremheaderfont{\normalfont\bfseries}
    \theorembodyfont{\upshape}

    \MakeMajorTheorem{example}[thm]{Example}
}
